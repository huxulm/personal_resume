% !TEX TS-program = xelatex
% !TEX encoding = UTF-8 Unicode
% !Mode:: "TeX:UTF-8"

\documentclass{resume}
\usepackage{zh_CN-Adobefonts_external} % Simplified Chinese Support using external fonts (./fonts/zh_CN-Adobe/)
%\usepackage{zh_CN-Adobefonts_internal} % Simplified Chinese Support using system fonts
\usepackage{linespacing_fix} % disable extra space before next section
\usepackage{cite}

\begin{document}
\pagenumbering{gobble} % suppress displaying page number

\name{许令明}

% {E-mail}{mobilephone}{homepage}
% be careful of _ in emaill address
\contactInfo{(+86) 18616777015}{huxulm@gmail.com}{后端研发工程师}{GitHub @huxulm}
% {E-mail}{mobilephone}
% keep the last empty braces!
%\contactInfo{xxx@yuanbin.me}{(+86) 131-221-87xxx}{}
 
\section{个人简介}
工作踏实负责,专注服务器(后端)技术研发,重视基础学习,关注新技术
% \section{\faGraduationCap\ 教育背景}
\section{教育背景}
\datedsubsection{\textbf{重庆大学},化学工程与工艺,\textit{工学学士}}{2011.9 - 2015.6}
% \section{\faCogs\ IT 技能}
\section{编程技能}
% increase linespacing [parsep=0.5ex]
\begin{itemize}[parsep=0.2ex]
  \item \textbf{编程语言}: Golang, Shell/Bash, Java, Python, JavaScript
  \item \textbf{数据库}: 关系型(MySQL、PG), KV类, NoSQL
  \item \textbf{操作系统}: Linux/Unix
  \item \textbf{其他}: 系统架构、Kubernetes、CI/CD(DevOps/GitOps)
\end{itemize}

% \end{itemize}

\section{工作经历}
\datedsubsection{\textbf{中科院新疆天文台 | XAO}  工程师}{2019.3-至今}
\begin{itemize}
  \item 参与射电天文望远镜控制系统服务端系统架构设计, 主要包括: OAuth2+LDAP开发认证系统、go-GRPC开发微服务核心组件、grpc2http网关设计开发; 搭建Kubernetes集群, 保障系统稳定运行; Gitlab+ArgoCD实现系统自动化持续集成部署
  \item 参与射电天文望远镜控制系统前端架构设计, 主要包括: 设计开发基于网页的控制前端(ReactJS)并协同认证系统与后端API实现望远镜远程能力
  \item kubernetes集群更新与维护, 基于Gitlab和ArgoCD在集群中搭建了一套高效的CI/CD系统
\end{itemize}

\datedsubsection{\textbf{实投(上海)互联网金融信息服务有限公司 }    系统架构工程师}{2017.1-2019.2}
\begin{itemize}
  \item 微服务化: 将原有业务系统拆分解耦,基于Dubbo框架进行微服务改造, 搭建容器化应用管理平台, 使用kubernetes对服务器集群管理
  \item 参与支付系统设计师开发: API设计、支付系统开放平台、文档编写
  \item 后台管理系统: 在原有系统业务基础上,重构了一套前后端分离的后台管理系统
\end{itemize}

\datedsubsection{\textbf{丰趣海淘网络科技有限公司}  JAVA研发工程师}{2016.6-2016.12}
\begin{itemize}
  \item 负责\textbf{工单系统(Ticket)} {Ticket是基于dubbo架构驱动的分布式微服务系统,主要用于帮助公司订单,供应链等系统收集业务异常,帮助业务快速定位解决问题。}
  \item 参与开发时效预警项目,用于解决供应链系统出库单状态流转时效全路由监控,统计。为业务人员分析系统时效提供便捷。
\end{itemize}

\datedsubsection{\textbf{上海华振物流有限公司}  JAVA研发工程师}{2015.6-2016.5}
\begin{itemize}
  \item 参与物流业务系统日常开发维护
\end{itemize}

%% Reference
%\newpage
%\bibliographystyle{IEEETran}
%\bibliography{mycite}
\end{document}
